\documentclass{article}
\usepackage{amsmath, amssymb, amsthm}
\usepackage[shortlabels]{enumitem}
\usepackage[utf8]{inputenc}

\title{Basics of Point-Set Topology}
\author{Stochastic Batman}
\date{January 19 - February 4, 2026}


\newcommand{\T}{\mathcal{T}}

\begin{document}
	
	\maketitle
	
	\section{Definitions before connectedness and compactness}
	
	\noindent \textbf{Topological Space:} For a set $X$, $(X, \T)$ is a \emph{topological space} if $\T$ (topology) is a collection of subsets of $X$ such that:
	\begin{enumerate}
		\item $\emptyset, X \in \T$.
		\item $\{U_i\}_{i=1}^n \in \T \implies \bigcap_{i=1}^n U_i \in \T$ (Finite intersection is closed).
		\item $\{U_\alpha\}_{\alpha \in A} \subseteq \T \implies \bigcup_{\alpha \in A} U_\alpha \in \T$ (Arbitrary union is closed).
	\end{enumerate}
	
	\vspace{0.5cm}
	
	\noindent \textbf{Continuity:} Let $(X,\T)$ and $(Y, \mathcal{S})$ be topological spaces. A function $f : X \to Y$ is \emph{continuous} if for every open set $V \in \mathcal{S}$ we have $f^{-1} (V) \in \T$.
	
	\vspace{0.5cm}
	
	\noindent \textbf{Homeomorphism:} Let $(X, \T)$ and $(Y, \mathcal{S})$ be topological spaces. A function $f: X \to Y$ is a homeomorphism if it is bijective; both $f$ and $f^{-1}$ are continuous.
	
	\vspace{0.5cm}
	
	\noindent \textbf{Basis $\mathcal{B}$ of Topological Space $(X, \T)$:} For every open set $U \in \T$, $\exists \mathcal{B}' \subseteq \mathcal{B}$ s.t. $U = \bigcup_{B \in \mathcal{B}'} B$.
	
	\vspace{0.5cm}
	
	\noindent \textbf{Second Countable:} A topological space $(X, \T)$ is second countable if it has a countable basis.
	
	\vspace{0.5cm}
	
	\noindent \textbf{Hausdorff Space:} $(X, \T)$ is Hausdorff space iff: $$\forall x, y \in X \, (x \neq y), \, \exists U, V \in \T : x \in U \land y \in V \land U \cap V = \emptyset$$
	
	\noindent \textbf{Locally Euclidean:} $(X, \T)$ is locally Euclidean of dimension $n$ if $\forall p \in X$ has a neighborhood $U$ that is homeomorphic to an open subset of $\mathbb{R}^n$ ($\forall p \in X, \exists \text{ open } U \ni p \text{ and } \phi: U \to V, \text{ where } V \subseteq \mathbb{R}^n \text{ is open}$).
	
	\vspace{0.5cm}
	
	\noindent \textbf{Convergence of a Sequence:} For $(X, \T)$, $\{x_n\}_{n \in \mathbb{N}}$ of points in $X$ converges to a point $x \in X$ if for every open set $U \in \T$ such that $x \in U, \exists N \in \mathbb{N}$ s.t. $\forall n \geq N, x_n \in U$.
	
	\vspace{0.5cm}
	
	\noindent \textbf{$n$-dimensional topological manifold:} A second countable Hausdorff space $(X, \T)$ that is locally Euclidean of dimension $n$.
	
	\vspace{0.5cm}
	
	\noindent \textbf{Upper half-space:} $\mathbb{R}^n \supseteq \mathbb{H}^n := \{ x \in \mathbb{R}^n \mid x_n \geq 0 \}$.
	
	\vspace{0.5cm}
	
	\noindent \textbf{$n$-dimensional manifold \underline{with boundary}:} A second countable Hausdorff space $(X, \T)$ in which every point has a neighborhood homeomorphic to an open subset of $\mathbb{R}^n$ or $\mathbb{H}^n$.
	
	\vspace{0.5cm}
	
	\noindent Let $S \subseteq X$. $\T_S := \{ U \subseteq S \mid U = S \cap V \text{ for some } V \in \T \}$ defines \textbf{subspace topology} on $S$. The pair $(S, \T_S)$ is a subspace.
	
	\vspace{0.5cm}
	
	\noindent Let $(X_1, \T_1), \dots, (X_n, \T_n)$ be topological spaces. \textbf{Product Topology} on $\prod_{i=1}^n X_i$ is generated by the basis $\mathcal{B} = \{ \prod_{i=1}^n U_i \mid \forall i, \, U_i \text{ is open in } \T_i \}$.
	
	\vspace{0.5cm}
	
	\noindent \textbf{Disjoint Union Spaces:} Let $\{(X_\alpha, \T_\alpha)\}_{\alpha \in A}$ be an indexed family of topological spaces, and let $\coprod \limits_{\alpha \in A} X_\alpha = \{ (x, \alpha) \mid x \in X_\alpha \}$ denote their disjoint union. Subset $U \subseteq \coprod \limits_{\alpha \in A} X_\alpha$ is open in disjoint union topology if $U \cap X_\alpha$ is open in $\T_\alpha, \forall \alpha \in A$. 
	
	One might also write a disjoint union of two sets with this symbol: $X \sqcup Y$.
	
	\vspace{0.5cm}
	
	\noindent \textbf{Quotient Topology:} Let $(X, \T_X)$ be a topological space and $Y$ a set. Let $q: X \to Y$ be surjective. Quotient topology on $Y$ via $q: \T_Y = \{ U \subseteq Y \mid q^{-1}(U) \in \T_X \}$ ($\T_Y$ is largest topology on $Y$ such that $q$ is continuous). 
	
	The definition of quotient topology might resemble the definition of continuity, but they differ: continuity is defined between two topological spaces and is a condition that a map must satisfy, while the quotient topology is defined from a topological space to a set - we construct a topology on that set so that the given surjection becomes continuous.
	
	\vspace{0.5cm}
	
	\noindent\textbf{Equivalence relation.} An equivalence relation $\sim$ on a set $X$ partitions $X$ into equivalence classes $[x]=\{y\in X: y\sim x\}$.
	
	\vspace{0.5cm}
	
	\noindent\textbf{Quotient space (set).} Given $\sim$ on $X$, define the quotient set
	$$ X/{\sim}=\{[x]\mid x\in X\} $$
	
	\vspace{0.5cm}
	
	\noindent\textbf{Wedge sum.} For a family of topological spaces $\{(X_i, \T_i)\}_{i\in I}$ with chosen basepoints $x_i\in X_i$,
	$$ \bigvee_{i\in I}X_i=\big(\coprod_{i\in I}X_i\big)/\sim, $$
	where all basepoints $x_i$ are identified (i.e., $x_i\sim x_j$ for all $i,j\in I$ and no other identifications).
	
	\vspace{0.5cm}
	
	Let \noindent $q: (X, \T) \to Y$ be any map. $q^{-1}(y)$ for $y \in Y$ is called a \textbf{fiber} of $q$. A subset $U \subseteq X$ is called \textbf{saturated} w.r.t. $q$, if $U = q^{-1}(V)$ for some $V \subseteq Y$.
	
	\vspace{0.5cm}
	
	\noindent Let $f: (X, \T) \to (Y, \mathcal{S})$ be a continuous map that is either open or closed.
	\begin{enumerate}[(a)]
		\item If $f$ is injective, it is a topological embedding.
		\item If $f$ is surjective, it is a quotient map.
		\item If $f$ is bijective, it is a homeomorphism.
	\end{enumerate}
	
	
	\noindent \textbf{Adjunction Spaces:} Let $(X, \T_X)$ and $(Y, \T_Y)$ be topological spaces, $A \subseteq X$ closed, $f: A \to Y$ a continuous function. Define adjunction space (where $a \sim f(a), \forall a \in A$): $$X \cup_f Y = X \sqcup  Y / \sim$$
	
	
	\section{Connectedness}
	
	\noindent \textbf{Disconnected:} A topological space $(X, \T)$ is disconnected iff it can be expressed as the disjoint union of two non-empty open subset $U \sqcup V = X$.
	
	\vspace{0.5cm}
	
	\noindent \textbf{Connected:} A topological space $(X, \T)$ that is NOT disconnected is said to be connected.
	
	\vspace{0.5cm}
	
	\noindent $(X, \T)$ is connected $\iff X, \emptyset$ are the only subsets of $X$ that are simultaneously open and closed (clopen for short).
	
	\noindent \textbf{Proof:} 
	
	\noindent $(\implies)$ $(X, \T)$ connected. Suppose $U \subseteq X$ is clopen. $U$ open and $U^c = X \setminus U$ open. Then $U \cup U^c = X \implies U = \emptyset$ or $U^c = \emptyset \implies U = X$.
	
	\noindent $(\impliedby)$ $X, \emptyset$ are the only clopen sets of $X$. If $X = U \cup V$, $U, V$ open $\implies U^c = X \setminus U = (U \cup V) \setminus U = V$ (so open)$\implies U$ clopen $\implies U = X$ or $U = \emptyset$ ($\implies V = \emptyset$). $\square$
	
	\vspace{0.5cm}
	
	\noindent \textbf{Path:} Let $(X, \T)$ be a topological space and $p, q \in X$. A \emph{path} in $X$ from $p$ to $q$ is a continuous map $f: [0, 1] \to X$ with $f(0) = p$ and $f(1) = q$.
	
	\vspace{0.5cm}
	
	\noindent \textbf{Path-connectedness:} A space $(X, \T)$ is called \emph{path-connected} iff for all $p, q \in X$, there is a path in $X$ from $p$ to $q$.
	
	\vspace{0.5cm}
	
	
	\noindent \textbf{Component:} Let $(X, \T)$ be a topological space. A \emph{component} of $(X, \T)$ is a maximal (to make sure subsets of a connected set do not count as separate components) nonempty connected subset of $X$.
	
	\vspace{0.5cm}
	
	\noindent \textbf{Path Component:} A path component of $(X, \T)$ is a maximal nonempty path-connected subset.
	
	\vspace{0.5cm}
	
	\noindent \textbf{Locally (Path) connected:} A topological space $(X, \T)$ is locally (path) connected if it admits a basis of (path) connected open subsets.
	
	\noindent This means any open neighborhood of $p \in X$ contains a (path) connected open set containing $p$.
	
	\section{Compactness}
	
	\noindent \textbf{Open Cover:} An \emph{open cover} of a space $X$ is a collection $\mathcal{U}$ of open subsets of $X$ whose union is $X$.
	
	\vspace{0.5cm}
	
	\noindent \textbf{Subcover:} A \emph{subcover} of $\mathcal{U}$ is a subcollection of elements of $\mathcal{U}$ that still covers $X$.
	
	\vspace{0.5cm}
	
	\noindent \textbf{Compact:} A space $X$ is \emph{compact} if every open cover of $X$ has a finite subcover.
	
	\vspace{0.5cm}
	
	\noindent A subset $A \subseteq X$ is compact if it is compact as a subspace.
	
	\vspace{0.5cm}
	
	\noindent \textbf{Compact Subspace Lemma:} A subset $A \subseteq X$ is compact iff every collection $\{U_\alpha\}$ of open subsets of $X$ with $\bigcup \limits_\alpha U_\alpha \supseteq A$ has a finite subcollection $\{U_{\alpha_k}\}_{k=1}^n$ satisfying $\bigcup \limits_{k=1}^n U_{\alpha_k} \supseteq A$.
	
	\vspace{0.5cm}
	
	\noindent Let $X$ and $Y$ be spaces and $f: X \to Y$ be continuous. If $X$ is compact, then so is $f(X)$.
	
	\vspace{0.5cm}
	
	\noindent Some compactness results:
	\begin{enumerate}[(a)]
		\item Closed subsets of compacts spaces are compact.
		\item Compact subsets of Hausdorff spaces are closed.
		\item Compact subsets of metric spaces are bounded.
		\item Finite products of compact spaces are compact.
		\item Quotients of compact spaces are compact.
	\end{enumerate}
	
	\vspace{0.5cm}
	
	\noindent \textbf{Closed and bounded intervals in $\mathbb{R}$ are compact}.

	\vspace{0.5cm}
	
	\noindent \textbf{Heine-Borel Theorem:} A subset $S \subseteq \mathbb{R}^n$ is compact iff it is closed and bounded.
	
	\vspace{0.5cm}
	
	\noindent \textbf{Extreme Value Theorem:} If $X$ is a compact space and $f: X \to \mathbb{R}$ continuous, then $f$ is bounded and attains its maximum and minimum values on $X$.
	
	\vspace{0.5cm}
	
	\noindent \textbf{Closed Map Lemma:} Suppose $f$ is continuous map from a compact space to a Hausdorff space. Then:
	\begin{enumerate}[(a)]
		\item $f$ is a closed map.
		\item $f$ injective $\implies f$ topological embedding. 
		\item $f$ surjective $\implies f$ quotient map.
		\item $f$ bijective $\implies f$ homeomorphism.  
	\end{enumerate}
	
	\section{Riemannian Geometry}
	
	\noindent \textbf{Chart:} Let $(M, \T)$ be a topological manifold. A \emph{chart} is a pair $(U, \phi)$ where $U \in \T$ is an open set and $\phi: U \to \widehat{U} \subseteq \mathbb{R}^n$ is a homeomorphism onto an open subset of $\mathbb{R}^n$.
	
	\vspace{0.5cm}
	
	\noindent \textbf{Transition Map:} Let $(U_\alpha, \phi_\alpha)$ and $(U_\beta, \phi_\beta)$ be charts with $U_\alpha \cap U_\beta \neq \emptyset$. The map $\tau_{\alpha, \beta} = \phi_\beta \circ \phi_\alpha^{-1} : \phi_\alpha(U_\alpha \cap U_\beta) \to \phi_\beta(U_\alpha \cap U_\beta)$ is called a \emph{transition map}.
	
	Unlike maps defined on the manifold, transition maps are functions between open subsets of Euclidean space $\mathbb{R}^n$, so we can use standard calculus to differentiate them.
	
	\vspace{0.5cm}
	
	\noindent \textbf{Atlas:} An \emph{atlas} $\mathcal{A}$ on $M$ is a collection of charts $\{(U_\alpha, \phi_\alpha)\}_{\alpha \in A}$ that covers $M$ (i.e., $\bigcup_{\alpha \in A} U_\alpha = M$).
	
	\vspace{0.5cm}
	
	\noindent \textbf{Smooth Structure:} An atlas is called \emph{smooth} if all its transition maps are $C^\infty$ (infinitely differentiable). A \emph{smooth structure} on $M$ is a maximal smooth atlas (one that contains every possible chart compatible with it).
	
	\vspace{0.5cm}
	
	\noindent \textbf{Smooth Manifold:} A topological manifold $(M, \T)$ is a \emph{smooth manifold} if it is equipped with a smooth structure.
	
	\vspace{0.5cm}
	
	\noindent \textbf{Tangent Space:} For a point $p$ in a smooth manifold $M$, the \emph{tangent space} $T_p M$ is the vector space consisting of all tangent vectors at $p$.
	\noindent (Intuitively, it is the space of all possible velocity vectors $\gamma'(0)$ of smooth curves $\gamma: (-\epsilon, \epsilon) \to M$ passing through $p$ at $t=0$).
	
	\vspace{0.5cm}
	
	\noindent \textbf{Riemannian Metric:} Let $M$ be a smooth manifold. A \emph{Riemannian metric} $g$ on $M$ is a field of inner products, assigning to each point $p \in M$ a positive-definite inner product $g_p: T_p M \times T_p M \to \mathbb{R}$ on the tangent space $T_p M$, varying smoothly with $p$.
	
	\vspace{0.5cm}
	
	\noindent \textbf{Riemannian Manifold:} A pair $(M, g)$ consisting of a smooth manifold $M$ and a Riemannian metric $g$.
	
	\vspace{0.5cm}
	
	\noindent \textbf{Geodesic:} A \emph{geodesic} is a smooth curve $\gamma: I \to M$ (where $I \subseteq \mathbb{R}$ is an interval) that is "straight" with respect to the metric $g$. Formally, it satisfies the geodesic equation $\nabla_{\dot{\gamma}}\dot{\gamma} = 0$ (zero acceleration).
	
	\vspace{0.5cm}
	
	\noindent \textbf{Maximal Geodesic:} A geodesic $\gamma: I \to M$ is \emph{maximal} if its domain $I$ cannot be extended to any larger interval $J \supsetneq I$ while preserving the geodesic property.
	
	\vspace{0.5cm}
	
	\noindent \textbf{Geodesic Completeness:} A Riemannian manifold $(M, g)$ is \emph{geodesically complete} if every maximal geodesic is defined on all of $\mathbb{R}$ (i.e., the domain is $(-\infty, \infty)$).
	
	\vspace{0.5cm}
	
	\noindent \textbf{Complete Riemannian Manifold:} A connected Riemannian manifold $(M, g)$ is called \emph{complete} if it is geodesically complete. 
	
	\noindent By the \textbf{Hopf-Rinow Theorem}, for a connected Riemannian manifold, the following are equivalent:
	\begin{enumerate}[(a)]
		\item $(M, g)$ is geodesically complete.
		\item The metric space $(M, d_g)$ induced by the Riemannian metric is complete (every Cauchy sequence converges).
		\item The closed and bounded subsets of $M$ are compact (Heine-Borel property).
	\end{enumerate}
	
\end{document}